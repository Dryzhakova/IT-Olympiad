\documentclass[12pt,a4paper]{article}
\usepackage[polish]{babel}
\usepackage[T1]{fontenc}
\usepackage[]{algorithm2e}
\usepackage{listings}

\usepackage{color}
\usepackage{listings}
\usepackage{graphicx}
\usepackage{subfigure}
\graphicspath{ {pictures/} }
\lstloadlanguages{% Check Dokumentation for further languages ...
	C,
	C++,
	csh,
	Java
}

\definecolor{red}{rgb}{0.6,0,0} % for strings
\definecolor{blue}{rgb}{0,0,0.6}
\definecolor{green}{rgb}{0,0.8,0}
\definecolor{cyan}{rgb}{0.0,0.6,0.6}

\lstset{
	language=csh,
	basicstyle=\footnotesize\ttfamily,
	numbers=left,
	numberstyle=\tiny,
	numbersep=5pt,
	tabsize=2,
	extendedchars=true,
	breaklines=true,
	frame=b,
	stringstyle=\color{blue}\ttfamily,
	showspaces=false,
	showtabs=false,
	xleftmargin=17pt,
	framexleftmargin=17pt,
	framexrightmargin=5pt,
	framexbottommargin=4pt,
	commentstyle=\color{green},
	morecomment=[l]{//}, %use comment-line-style!
	morecomment=[s]{/*}{*/}, %for multiline comments
	showstringspaces=false,
	morekeywords={ abstract, event, new, struct,
		as, explicit, null, switch,
		base, extern, object, this,
		bool, false, operator, throw,
		break, finally, out, true,
		byte, fixed, override, try,
		case, float, params, typeof,
		catch, for, private, uint,
		char, foreach, protected, ulong,
		checked, goto, public, unchecked,
		class, if, readonly, unsafe,
		const, implicit, ref, ushort,
		continue, in, return, using,
		decimal, int, sbyte, virtual,
		default, interface, sealed, volatile,
		delegate, internal, short, void,
		do, is, sizeof, while,
		double, lock, stackalloc,
		else, long, static,
		enum, namespace, string},
	keywordstyle=\color{cyan},
	identifierstyle=\color{red},
}
\usepackage{caption}
\DeclareCaptionFont{white}{\color{white}}
\DeclareCaptionFormat{listing}{\colorbox{blue}{\parbox{\textwidth}{\hspace{15pt}#1#2#3}}}
\captionsetup[lstlisting]{format=listing,labelfont=white,textfont=white, singlelinecheck=false, margin=0pt, font={bf,footnotesize}}


\addtolength{\hoffset}{-1.5cm}
\addtolength{\marginparwidth}{-1.5cm}
\addtolength{\textwidth}{3cm}
\addtolength{\voffset}{-1cm}
\addtolength{\textheight}{2.5cm}
\setlength{\topmargin}{0cm}
\setlength{\headheight}{0cm}

\begin{document}
	
	\title{Języki Skrypowe\\\small{dokumentacja projektu Antysymetria}
	\begin{center}
 		\begin{large}
 		Politechnika Śląska\\Matematyka Stosowana, Informatyka\\Rok II, Semestr III, 2022/2023\\Grupa 1/2					\author{Kateryna Dryzhakova}
 		\end{large}
	\end{center}
	\date{\today}}

	\maketitle
	\newpage
	\section*{Część I}
	\subsection*{Opis programu}

	Bajtazar studiuje różne napisy złożone z zer i jedynek. Niech  będzie takim napisem, przez I(R) będziemy oznaczać odwrócony(czyli "czytany wspak") napis I, a przez U będziemy oznaczać napis powstały z  przez zamianę wszystkich zer na jedynki, a jedynek na zera.

Bajtazara interesuje antysymetria, natomiast niezbyt lubi wszystko co symetryczne. Antysymetria nie jest tylko prostym zaprzeczeniem symetrii. Powiemy, że (niepusty) napis I jest antysymetryczny, jeżeli dla każdej pozycji i w I, i-ty znak od końca jest różny od i-tego znaku, licząc od początku. W szczególności, niepusty napis  złożony z zer i jedynek jest antysymetryczny wtedy i tylko wtedy, gdy I = U(R). Na przykład, napisy 00001111 i 010101 są antysymetryczne, natomiast 1001 nie jest.

W zadanym napisie złożonym z zer i jedynek chcielibyśmy wyznaczyć liczbę jego spójnych (tj. jednokawałkowych) niepustych fragmentów, które są antysymetryczne. Jeżeli różne fragmenty odpowiadają takim samym słowom, to i tak należy je policzyć wielokrotnie.

Wejście
Pierwszy wiersz standardowego wejścia zawiera liczbę n, oznaczającą długość napisu. Drugi wiersz zawiera napis złożony z liter 0 i/lub 1 o długości n. Napis ten nie zawiera żadnych odstępów.

Wyjście
Pierwszy i jedyny wiersz standardowego wyjścia powinien zawierać jedną liczbę całkowitą, oznaczającą liczbę spójnych fragmentów wczytanego napisu, które są antysymetryczne.

	\subsection*{Instrukcja obsługi}

	Aby uruchomić program należy włączyć skrypt menu.bat otwierający menu obsługi naszego programu.
 Po uruchomieniu wyświetli nam się tekst z instrukcją obsługi programu,wymagający podania przez użytkownika liczby w celu wykonania odpowiadającej mu funkcji.
 
 \begin{figure}[h]
    \centering
    \includegraphics[width=0.5\textwidth]{choice}
    \caption{Główne menu programu}
\end{figure}



Możliwe wybory są następujące:
\\1. Start the program - Uruchamia program pobierając przy tym wszystkie dane prez użytkownika i tworzy raport.html.

 \begin{figure}[h]
    \centering
    \includegraphics[width=0.9\textwidth]{task1}
    \caption{ Przykładowy komunikat o pomyślnej próbie uruchomienia programu}
\end{figure}

 \begin{figure}[h]
    \centering
    \includegraphics[width=0.9\textwidth]{raport}
    \caption{Przykładowy raport programu}
\end{figure}

2. Program information - Wypisuje na ekranie konsoli opis założeń programu

 \begin{figure}[!h]
    \centering
    \includegraphics[width=1\textwidth]{information}
    \caption{Przykładowy raport programu}
\end{figure}

3. Backup - Tworzy kopię zapasową danych w katalogu backups zawierającą raport.html oraz zawartość folderów input i output

\begin{figure}[htp]
    \centering
    \includegraphics[width=1\textwidth]{cmdRaport}
    \caption{Przykładowy raport programu}
\end{figure}

\begin{figure}[htp]
    \centering
    \includegraphics[width=0.5\textwidth]{treeBackups}
    \caption{Przykładowy raport programu}
\end{figure}

4. Zakoncz - Zamyka menu, kończąc tym samym program.

	Podanie innej liczby lub znaku, skutkuje powiadomieniem o wprowadzeniu niepoprawnego
polecenia

Struktura danych programu:

Program składa się z następującej struktury danych,wymaganych do prawidłowego uruchomienia aplikacji:

- menu.bat - Skrypt batch będący menu, którym uruchamia się program, wyświetla informacje o programie jak i tworzy kopie zapasową danych otrzymanych w wyniku wykonania tegoż programu.

- antysymetria.py - Skrypt python zawierający główny program, pobierający pliki wejściowe zawierające wpisane przez użytkownika liczby i tworzący plik wyjścia zawierający liczbu fragmentów antysymetrycznych.

- raport.py - Skrypt python pobierający dane z plików wejścia oraz wyjścia i generujący plik raport.html zawierający raport wszystkich danych w postaci tabeli.

- Katalog input - pliki wejściowe zawierające zapisane dane przez użytkownika, nazwane według klucza:
\begin{equation}
     input0.txt
\end{equation}

Ponadto program w wyniku działania tworzy dodatkowo katalogi output, backups oraz plik raport.html\\

\begin{figure}[ht!]  
\vspace{-4ex} \centering \subfigure[]{
\includegraphics[width=0.23\linewidth]{tree1} \label{fig:actuatorscouplingSheme_decoupledcase} }  
\hspace{4ex}
\subfigure[]{
\includegraphics[width=0.22\linewidth]{tree2} \label{fig:actuatorscouplingSheme_nearestcoupledcase} }
\hspace{4ex}
\subfigure[]{ \includegraphics[width=0.24\linewidth]{tree3} \label{fig:actuatorscouplingSheme_nearestcoupled_and_diag_case} }  
\caption{ Struktura danych programu w formie drzwa} 
\end{figure}


	\section*{Część II}
	\subsection*{Opis działania} 
Skrypt menu.sh pobiera liczby prez użytkownika i przekazuje pojedynczo jako argument do antysymetria.py. Następnie dla pobranej liczby program utworzy pętlę, która na początku sprawdzić czy podana długosc zbiega się z dłgosci liczby.

Następnie, przejdzie przez każdy element napisu i sprawdzi, czy wybrany element oraz jego odpowiednik (element, który jest równy liczbie elementów napisu minus indeks elementu plus jeden) są różne. Jeśli tak, to należy zwiększyć licznik antysymetrycznych fragmentów o jeden.

Dalej należy znaleźć wszystkie spójne fragmenty napisu, które są antysymetryczne. Można to zrobić, tworząc pętlę, która przejdzie przez każdy element napisu i sprawdzi, czy jego odpowiednik jest różny od niego. Jeśli tak, to sprawdzamy, czy fragment napisu od obecnego elementu do jego odpowiednika jest antysymetryczny. Jeśli tak, to zwiększamy licznik antysymetrycznych fragmentów o jeden.

Finalnie program kolejno wypisuje zawartosc znalezionych fragmentów antysymetrycznych do pliku output.
Otrzymane w ten sposób wyniki są następnie przetwarzane przez raport.py, który tworzy plik raport.html, który w tabeli umieszcza zarówno zawartość plików input oraz output, tylko
i wyłącznie jeśli numer w ich nazwie jest taki sam, co sprawia, że raport zawiera jedynie pomyślnie wykonane iteracje programu. Ostatecznie uruchamiana jest domyślna przeglądarka użytkownika, w której wyświetla się
utworzony raport.html.
	
    
	\subsection*{Algorytmy}
	
	\begin{algorithm}[H]
		\KwData{Dane wejściowe pobrane przez uzytkownika $dataInput,length$}
		\KwResult{Dane wyjściowe plik output}
		\While{$dataInput != length$}{
			Enter another correct number\;
			}
		
		 $amountAntisymmetricsFragments(dataInput, output)$
		
		Function amountAntisymmetricsFragments(dataInput, output)\\
		\For{$i<dataInput.len$}{
			\For{$i+1<dataInput.len+1$}{
				\If{$antisymmetrics(dataInput[i:j])$}{
					Dodaj do output;
					Obliczyć liczbę wchodzenij;
				}
			}
		}
		\Return $liczbaWchodzenij$;
		
    	Function antisymmetrics(dataInput)\\
		\For{$i<dataInput.len$}{
			\If{$dataInput[i]<dataInput[-i-1]$}{
				\Return $False$;
			}
			\Return $True$;
		}
		Przypisz do $output$ zawartość function $amountAntisymmetricsFragments$
		\caption{Algorytm wyszukiwania antysymetrii liczby}
	\end{algorithm}
	
	\begin{figure}[htp]
    \centering
    \includegraphics[width=0.7\textwidth]{Dryzhakova_Projekt}
    \caption{Schemat blokowy}
\end{figure}
	

	\subsection*{Implementacja systemu}
	Uruchomienie programu z poziomu skryptu menu.bat powoduje sprawdzenie czy istnieje katalog na pliki wynikowe "output", jeśli tak to usuwa go wraz z zawartością oraz tworzy go
na nowo, następnie uruchamia antysymetria.py wraz z argumentem zadanym przez użytkownika. Jeśli program znalazł wynik i zakończył się powodzeniem, tworzy on
plik wynikowy do katalogu "output". Sprawdzane jest czy w katalogu "output"znajdują się jakiekolwiek pliki. Jeśli tak to uruchamiany jest skrypt raport.py który tworzy plik raport.html w głównym katalogu projektu, a jeśli nie to program kończy się z odpowiednim komunikatem.

Opcja wyświetlająca informacje wypisuje dane z pliku "information.txt"znajdującego się w katalogu "mój projekt".

Wybranie funkcji backup sprawdza czy istnieje w głównym katalogu plik raport.html, jeśli tak to tworzy katalog "backups" o ile takowy juz nie istnieje, następnie tworzy katalog nazwany aktualną datą i godziną wewnątrz folderu "backups", który natomiast zawiera kopię całego stworżonego
katalogu "input", katalogu "output" oraz pliku raport.html.\\\\

\textbf{Wykorzystane biblioteki i przykłady ich użycia}\\
\textbf{os}
	\begin{lstlisting}
		for f in os.listdir(dir):
   			os.remove(os.path.join(dir, f))
   		//sprawdza czy plik istnieje oraz usuwa go
	\end{lstlisting}
	\begin{lstlisting}
		os.chdir("input")
		//change the current directory to input directory
	\end{lstlisting}
\textbf{sys}
\begin{lstlisting}
		with open(sys.argv[1],"a+") as g:
    		g.write(str(data_input))
    	//zapisuje podanu liczbu do pliku
	\end{lstlisting}
	\subsection*{Testy}
Dane wejściowe:
Długość 8
Liczba 11001011

Wypisyemy antysymetryczne fragmenty: 
01 (pojawia się dwukrotnie), 10 (także dwukrotnie), 0101, 1100 oraz 001011.

Dane wyjściowe: 
7

\begin{figure}[htp]
    \centering
    \includegraphics[width=0.7\textwidth]{11001011}
    \caption{Wynik programu w konsoli}
\end{figure}

Dane wejściowe:
Długość 3
Liczba 11001011

Dane wyjściowe: 
Powtórz wpisanie liczby

\begin{figure}[htp]
    \centering
    \includegraphics[width=0.7\textwidth]{101}
    \caption{Wynik programu w konsoli}
\end{figure}

Długość 4
Liczba 1100

Wypisyemy antysymetryczne fragmenty: 
1100, 10.

Dane wyjściowe: 
2

\begin{figure}[htp]
    \centering
    \includegraphics[width=0.6\textwidth]{1100}
    \caption{Wynik programu w konsoli}
\end{figure}


	\newpage
	\section*{Pełen kod aplikacji}
\textbf{menu.bat}
\begin{lstlisting}
@echo off

rem set choice_from_cli = %1
echo Antysymetria Kateryna Dryzhakova
:menu
echo  ---------------------------
echo             Menu             
echo  ----------------------------
echo.                             
echo   1. Start the program       
echo   2. Program information     
echo   3. Backup                  
echo   4. Exit                    
echo  ----------------------------
echo.
set /p choice=Your choice(1-4) 

if %choice%==1 goto task1
if %choice%==2 goto task2
if %choice%==3 goto task3
if %choice%==4 goto exit
echo Your choice is not included in the range 1-4
goto menu
:task1
echo.
IF EXIST raport.html DEL raport.html
IF NOT EXIST output mkdir output
echo "<HTML>" >> raport.html
DEL /Q output
for /f "delims=" %%a in ('dir /b input') do (
    call python antysymetria.py %%a
)
call python raport.py
echo.
goto menu
:task2
echo.
chcp 1251
type information.txt
pause
echo.
goto menu
:task3
echo.
IF NOT EXIST backups mkdir backups
set name=%date%--%TIME:~1,7%
set name=%name::=-%
IF EXIST raport.html mkdir backups\%name%
robocopy input backups\%name%\input
robocopy output backups\%name%\output
copy raport.html backups\%name%\raport.html
echo.
goto menu
:exit
\end{lstlisting}

\textbf{antysymetria.py}
\begin{lstlisting}
import sys
import os

def antisymmetrics(data_input):
    for i in range(len(data_input)):
        if data_input[i] == data_input[-i-1]:
            return False
    return True

def amount_antisymmetrics_fragments(data_input, output):
    count = 0
    for i in range(len(data_input)):
        for j in range(i+1, len(data_input)+1):
            # sprawdzamy, czy podnapis o indeksach (i, j) jest antysymetryczny
            if antisymmetrics(data_input[i:j]):
                output += data_input[i:j]
                output += " "
                count += 1
    return count

dir = D:\university\Politechnika\Jezyki skryptowe\Projekt\Moj projekt\input
for f in os.listdir(dir):
    os.remove(os.path.join(dir, f))
	
length = int(input("Enter the length of the number > "))
data_input = input("Enter your number > ")

while(len(data_input) != length):
	data_input = input("The string you entered is not the correct length. Please, try again > ")
	
os.chdir("input")

with open(sys.argv[1],"a+") as g:
    g.write(str(data_input))

output = []

amount = amount_antisymmetrics_fragments(data_input, output)
print("Amount of antisymmetrics fragments > ",amount)
print("Antisymmetric fragments > ", *output)

os.chdir("..")
os.chdir("output")
with open(sys.argv[1],"a+") as g:
    g.write(str(amount))	
\end{lstlisting}

\textbf{raport.py}
\begin{lstlisting}
import os
from datetime import date, datetime
from os.path import isfile, join, exists

now = datetime.now()
fulldate = now.strftime("%d-%m-%Y %H:%M:%S")

outputfile = open("Raport.html", "w")

outputfile.write(f"""
<html>
  <head>
    <title>Raport Antysymetria</title>
  </head>
  <body>
  <h1>Raport {fulldate}</h1>
  <table>
    <tr>
      <th>input</th>
      <th>output</th>
    <tr>
    """)
	

inputfiles = [file for file in os.listdir("input") if isfile(join("input", file))]
for x in range(len(inputfiles)):
    with open(f"input/{x}.txt", "r") as f:
        lines = [str(line.rstrip()) for line in f]
    outputfile.write(""" <tr><td>""" + str(lines).replace("'", "")[1:-1])

    outputfile.write("""</td>
          <td>""")
    with open(f"output/{x}.txt", "r") as f:
        lines = [str(line.rstrip()) for line in f]
    outputfile.write(str(lines).replace("'", "")[1:-1])
    outputfile.write("""</td>
        </tr>""")
outputfile.write("""
  </table>
  </body>
</html>""")

outputfile.close()
\end{lstlisting}

\end{document}